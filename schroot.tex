\documentclass[a4paper]{article}
\usepackage{verbatim}

\begin{document}
%%%%%%%%%%%%%%%%%%%%%%%%%%%%%%%%%%%%%%%%%%%%%%%%%%%%%%%%%%%%%%%%%%%%%%%%%%%%%%
%%%%%%%%%%%%%%%%%%%%%%%%%%%%%%%%%%%%%%%%%%%%%%%%%%%%%%%%%%%%%%%%%%%%%%%%%%%%%%
%%%
%%%  Preparation
%%%
%%%%%%%%%%%%%%%%%%%%%%%%%%%%%%%%%%%%%%%%%%%%%%%%%%%%%%%%%%%%%%%%%%%%%%%%%%%%%%
%%%%%%%%%%%%%%%%%%%%%%%%%%%%%%%%%%%%%%%%%%%%%%%%%%%%%%%%%%%%%%%%%%%%%%%%%%%%%%
\section{Preparation}

Install the following

\begin{verbatim}
sudo apt-get install schroot debootstrap
\end{verbatim}

%%%%%%%%%%%%%%%%%%%%%%%%%%%%%%%%%%%%%%%%%%%%%%%%%%%%%%%%%%%%%%%%%%%%%%%%%%%%%%
%%%%%%%%%%%%%%%%%%%%%%%%%%%%%%%%%%%%%%%%%%%%%%%%%%%%%%%%%%%%%%%%%%%%%%%%%%%%%%
%%%
%%%  The master
%%%
%%%%%%%%%%%%%%%%%%%%%%%%%%%%%%%%%%%%%%%%%%%%%%%%%%%%%%%%%%%%%%%%%%%%%%%%%%%%%%
%%%%%%%%%%%%%%%%%%%%%%%%%%%%%%%%%%%%%%%%%%%%%%%%%%%%%%%%%%%%%%%%%%%%%%%%%%%%%%

\section{Creating the master}
\subsection{Building precise}
First we don't want anything from the outside world going into the master.
Assume we are in a folder 'fakeroots', create a folder for the master

\begin{verbatim}
cd fakeroots
mkdir master
\end{verbatim}

Now install precise into this folder

\begin{verbatim}
sudo debootstrap --variant=buildd --arch=amd64 precise 
   path2master/master 
   http://archive.ubuntu.com/ubuntu
\end{verbatim}

(all in one line)
Replace precise by another version if needed.

Go into the folder

\begin{verbatim}
cd path2master/master/etc/apt
\end{verbatim}

and modify the sources.list according to your needs. In my case I used

\begin{verbatim}
http://repogen.simplylinux.ch/
\end{verbatim}

\subsection{Creating the master profile}

Start by duplicating the default profile

\begin{verbatim}
cd /etc/schroot
sudo mkdir master
sudo cp default/* master
\end{verbatim}

Now we need to modify the profile for not getting some files from the outside

\begin{verbatim}
cd /etc/schroot/master
sudo vim copyfiles
sudo vim nssdatabases
sudo vim fstab
\end{verbatim}

Comment out all lines in these files, except the line with /proc in fstab. Then schroot won't copy the passwd file and /proc won't be mounted from the outside.

Modify the configuration file to have the correct paths

\begin{verbatim}
sudo vim /etc/schroot/master/config
\end{verbatim}

it should look like

\begin{verbatim}
# Settings for "master" profile.
FSTAB="/etc/schroot/master/fstab"
COPYFILES="/etc/schroot/master/copyfiles"
NSSDATABASES="/etc/schroot/master/nssdatabases"
\end{verbatim}

Next I created the configuration file

\begin{verbatim}
cd /etc/schroot/chroot.d
touch master.conf
\end{verbatim}

And filled it with the following content

\begin{verbatim}
[master]
description=Ubuntu precise pangolin master chroot
directory=path2master/master
root-users=lars <- your username here.
type=directory
users=m2user
script-config=master/config <- total path doesn't work
\end{verbatim}

Note that apparently all the users mentioned here have to exist on the outside as well.
We definetely have to modify the passwd file, since at this point it is a copy from the outside.

\subsection{Installing M2 and other things}

Move the M2 sources to a location (for example /home) inside the chroot:

\begin{verbatim}
sudo cp M2-1.4.0.1.tar.gz path2master/master/home/
\end{verbatim}

Now start the master schroot

\begin{verbatim}
schroot -c master -u root
\end{verbatim}

NO sudo in front of this command. Once inside, install everything you need using apt.
Note that you might get an error message depending on whether the folder you start the schroot in also exists in the schroot environment or not.

Inside the chroot

\begin{verbatim}
cd /home
tar -xzf M2-1.4.0.1.tar.gz
cd /
mkdir M2
mv /home/installed/* M2
rm -rf /home/installed/
rm /home/M2-1.4.0.1.tar.gz
apt-get update
apt-get upgrade
apt-get install vim
vim /etc/profile
\end{verbatim}

At the end of this file add the line

\begin{verbatim}
PATH=$PATH:/M2/bin
\end{verbatim}

Close the file and the schroot (Ctrl+d). You can restart the schroot and check the PATH variable for correctness.

Install all packages needed for M2. Start the schroot as root and inside

\begin{verbatim}
apt-get install libxml2
M2
\end{verbatim}

%%%%%%%%%%%%%%%%%%%%%%%%%%%%%%%%%%%%%%%%%%%%%%%%%%%%%%%%%%%%%%%%%%%%%%%%%%%%%%
%%%%%%%%%%%%%%%%%%%%%%%%%%%%%%%%%%%%%%%%%%%%%%%%%%%%%%%%%%%%%%%%%%%%%%%%%%%%%%
%%%
%%%  The Clone
%%%
%%%%%%%%%%%%%%%%%%%%%%%%%%%%%%%%%%%%%%%%%%%%%%%%%%%%%%%%%%%%%%%%%%%%%%%%%%%%%%
%%%%%%%%%%%%%%%%%%%%%%%%%%%%%%%%%%%%%%%%%%%%%%%%%%%%%%%%%%%%%%%%%%%%%%%%%%%%%%
\section{Cloning the master schroot}

We want to prepare a profile for the clones

\begin{verbatim}
cd /etc/schroot
sudo mkdir clone
sudo cp master/* clone
cd clone
sudo vim config
\end{verbatim}

edit this to look like

\begin{verbatim}
# Settings for "clone" profile.
FSTAB="/etc/schroot/clone/fstab"
COPYFILES="/etc/schroot/clone/copyfiles"
NSSDATABASES="/etc/schroot/clone/nssdatabases"
\end{verbatim}

\subsection{Readonly stuff from master}

Edit the fstab to fetch the master files

\begin{verbatim}
sudo vim /etc/schroot/clone/fstab
\end{verbatim}

it should look like

\begin{verbatim}
# fstab: static file system information for chroots.
# <file system>   <mount point>  <type>   <options>   <dump>   <pass>
/proc      /proc    none    bind        0       0
/proc      /proc    none    remount,bind,ro        0       0
path2master/master/bin /bin  none  bind     0  0
path2master/master/bin /bin  none  remount,bind,ro      0  0
path2master/master/usr /usr  none  bind     0  0
path2master/master/usr /usr  none  remount,bind,ro      0  0
path2master/master/srv /srv  none  bind     0  0
path2master/master/srv /srv  none  remount,bind,ro      0  0
path2master/master/lib /lib  none  bind     0  0
path2master/master/lib /lib  none  remount,bind,ro      0  0
path2master/master/sys /sys  none  bind     0  0
path2master/master/sys /sys  none  remount,bind,ro      0  0
path2master/master/boot   /boot none  bind     0  0
path2master/master/boot   /boot none  remount,bind,ro      0  0
path2master/master/run /run  none  bind     0  0
path2master/master/run /run  none  remount,bind,ro      0  0
path2master/master/M2 /M2  none  bind     0  0
path2master/master/M2 /M2  none  remount,bind,ro      0  0
\end{verbatim}

\subsection{Configuring the clone}

Create a folder for the clone (maybe inside fakeroots)

\begin{verbatim}
cd fakeroots
mkdir clone
\end{verbatim}

Edit the clone chroot config

\begin{verbatim}
cd /etc/schroot/chroot.d
sudo vim clone.conf
\end{verbatim}

to look like

\begin{verbatim}
[clone]
description=Ubuntu precise pangolin clone chroot
directory=path2clone/clone
root-users=lars
type=directory
users=m2user
script-config=clone/config
\end{verbatim}

\subsection{First test of the clone}

The clone will also need other files. So first I copy the complete master directory, then I remove certain folders

\begin{verbatim}
cd fakeroots
sudo cp -rf master/* clone
cd clone
sudo rm -rf bin sbin usr srv proc sys boot run M2
\end{verbatim}

Start the clone and check it

\begin{verbatim}
schroot -c clone -u root
touch /bin/bla
\end{verbatim}

should yield

\begin{verbatim}
touch: cannot touch `/bin/bla': Read-only file system
\end{verbatim}

\subsection{todo}

Store the configuration files in a location such that copyfiles fetches them every time we start. I.e. also modify copyfiles accordingly.

%%%%%%%%%%%%%%%%%%%%%%%%%%%%%%%%%%%%%%%%%%%%%%%%%%%%%%%%%%%%%%%%%%%%%%%%%%%%%%
%%%%%%%%%%%%%%%%%%%%%%%%%%%%%%%%%%%%%%%%%%%%%%%%%%%%%%%%%%%%%%%%%%%%%%%%%%%%%%
%%%
%%%  Schroot Commands
%%%
%%%%%%%%%%%%%%%%%%%%%%%%%%%%%%%%%%%%%%%%%%%%%%%%%%%%%%%%%%%%%%%%%%%%%%%%%%%%%%
%%%%%%%%%%%%%%%%%%%%%%%%%%%%%%%%%%%%%%%%%%%%%%%%%%%%%%%%%%%%%%%%%%%%%%%%%%%%%%
\section{Schroot Commands}
One can list all schroots with

\begin{verbatim}
schroot -l
\end{verbatim}

Kill all schroots with

\begin{verbatim}
schroot -e --all-sessions
\end{verbatim}

%%%%%%%%%%%%%%%%%%%%%%%%%%%%%%%%%%%%%%%%%%%%%%%%%%%%%%%%%%%%%%%%%%%%%%%%%%%%%%
%%%%%%%%%%%%%%%%%%%%%%%%%%%%%%%%%%%%%%%%%%%%%%%%%%%%%%%%%%%%%%%%%%%%%%%%%%%%%%
%%%
%%%  Possible Traps
%%%
%%%%%%%%%%%%%%%%%%%%%%%%%%%%%%%%%%%%%%%%%%%%%%%%%%%%%%%%%%%%%%%%%%%%%%%%%%%%%%
%%%%%%%%%%%%%%%%%%%%%%%%%%%%%%%%%%%%%%%%%%%%%%%%%%%%%%%%%%%%%%%%%%%%%%%%%%%%%%
\section{Possible Traps}
\begin{itemize}
\item Look at /etc/passwd. For some reason all users of the top system appear here. This might be good for others but not for us.

One can resolve this by manipulating the files in /etc/schroot/default. In this case the nssdatabases file is the one to be manipulated.
\item The shell via ssh looks weird.
\item We want different ports for all schroots, but we only have one file we can change.
\end{itemize}

\section{Things to delete}
\begin{itemize}
\item[sbin] ctrlaltdel, shutdown, reboot, restart, fdisk. Maybe don't mount this folder at all.
\item[/usr/sbin] Maybe don't mount this folder at all.
\end{itemize}
\end{document}
