\documentclass[a4paper]{article}
\usepackage{verbatim}

\begin{document}
\section{sudo}

First one needs to install the port sudo. Login as root, then

\begin{verbatim}
cd /usr/ports/security/sudo
make
make install clean
\end{verbatim}

Then one can select which group should contain the sudoers

\begin{verbatim}
visudo
\end{verbatim}

Maybe one needs to create that group and add users to it

\begin{verbatim}
pw groupadd group
pw groupmod group -m new_member
\end{verbatim}

\section{portmaster}
Install portmaster

\begin{verbatim}
cd /usr/ports/ports-mgmt/portmaster
sudo make install clean
\end{verbatim}

Now e.g. vim can be installed like

\begin{verbatim}
cd /usr/ports
sudo portmaster editors/vim
\end{verbatim}

\section{jails}
For virtualbox I inserted a second harddisc after installing freebsd on the first one.
Add the line

\begin{verbatim}
zfs_enable="YES"
\end{verbatim}

to the rc.conf.
Let ada1 be the new disc, now do

\begin{verbatim}
sudo gpart create -s gpt ada1
sudo gpart add -t freebsd-zfs ada1
sudo zpool create tank ada1p1
sudo zfs create -o mountpoint=/jails tank/jails
sudo zfs create -o mountpoint=/jails/public tank/jails/public
zfs list
\end{verbatim}

\section{Macaulay2}
Install Macaulay2

\begin{verbatim}
# inside ~/src
  svn co svn://svn.macaulay2.com/Macaulay2/trunk M2-trunk
  cd M2-trunk/M2
  gmake

\end{verbatim}

\end{document}
